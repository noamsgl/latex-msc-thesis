\usepackage{bbm} %for indicator
\usepackage{bm} %bold math stuff

% #################### CUSTOM COMMANDS ####################
\newcommand\blankPage{
\newpage
\mbox{}
\newpage
}

\newcommand\blankPageWithoutFooter{
\newpage
\mbox{}
\thispagestyle{empty} %no pagemark footer here
\newpage
}

\newcommand\blankPageWithoutHeader{
\newpage
\mbox{}
\thispagestyle{plain} %no header
\newpage
}

\newcommand\setPageStyle{
\pagestyle{scrheadings} %custom header/footer

\ifPRINT
    \ohead{\leftmark}
    \chead{}
    \ihead{\rightmark}
    \ofoot[\pagemark]{\pagemark}
    \cfoot[]{}
    \ifoot[]{}
\else
    \lohead{\leftmark}
    \cohead{}
    \rohead{\rightmark}
    \lehead{\leftmark}
    \cehead{}
    \rehead{\rightmark}
    \refoot[\pagemark]{\pagemark}
    \cefoot[]{}
    \lefoot[]{}
    \rofoot[\pagemark]{\pagemark}
    \cofoot[]{}
    \lofoot[]{}
\fi
\automark[section]{chapter}
}

% #################### abbreviations ####################

 %--------------------------------------------
% Add a period to the end of an abbreviation unless there's one
% already, then \xspace.
% \DeclareRobustCommand\onedot{\futurelet\@let@token\@onedot}
% \def\@onedot{\ifx\@let@token.\else.\null\fi\xspace}
% \def\@onedot{\xspace}

\def\eg{\emph{e.g.}~}  %for example, such as
\def\ie{\emph{i.e.,~}} %in other words
% \def\cf{\emph{cf.}~} \def\Cf{\emph{Cf.}~}
\def\etc{\emph{etc.}~} \def\vs{\emph{vs.}~}
\def\wrt{w.r.t.~} 
% \def\wrt{with respect to }  % The extra space at the end means \wrt blah would 
% not end up as ``with respect toblah''. 
\def\aka{a.k.a.~} 

% \def\dof{d.o.f.~}
% \def\etal{\emph{et al.}~}
\def\etal{\emph{et al.}~}
% \def\etal{\emph{et al.}~}
%
\def\iid{\emph{i.i.d.}~}
% \def\pdf{\emph{p.d.f.}~}

\def\rhs{r.h.s.~}
\def\lhs{l.h.s.~}

\newcommand{\supth}{\ensuremath{^{\text{th}}}} %\th was already occupied, usage: e.g. for k^th

%--------------------------------------------------
\def\naive{naïve~}
\def\Naive{Naïve~}
% TODO: add Matern

% \newcommand{\FIG}{Fig.~}
% \newcommand{\FIGS}{Figs.~}
% \newcommand{\EQN}{Eq.~}
% % \newcommand{\EQN}{Equation }
% \newcommand{\EQNS}{Eqs.~}
% \newcommand{\EQNS}{Equations }

%\newcommand{\SEC}{Sec.~}.  Use \autoref or ~\autoref instead.

% #################### Math convenience ####################

%https://tex.stackexchange.com/a/5255/199031
% \DeclareMathOperator*{\argmax}{arg\,max}
% \DeclareMathOperator*{\argmin}{arg\,min}

%About \ensuremath{#1}:
% The aim of it to allow #1 to be used in math mode and outside.
% yet, as we can see in this link, there sometimes may be problems with spacing when outside of mathmode.
% My take is use it only when you have a single letter, \ie \newcommand{\ZZ}{\ensuremath{\mathbb{Z}}}
% ans sparingly in other cases
%https://tex.stackexchange.com/questions/34830/when-not-to-use-ensuremath-for-math-macro

% probability
\newcommand{\given}{\ensuremath{\;\middle\vert\;}} %has to be inside \left \right
\newcommand{\prob}{\mathbb{P}}
% integers
\newcommand{\ZZ}{\ensuremath{\mathbb{Z}}}
\newcommand{\RR}{\ensuremath{\mathbb{R}}}
% \newcommand{\Rtwo}{\ensuremath{\RR^2}}
% \newcommand{\Rthree}{\ensuremath{\RR^3}}
% \newcommand{\Rsix}{\ensuremath{\RR^6}}
\newcommand{\Rn}{\ensuremath{\RR^n}}
\newcommand{\Rd}{\ensuremath{\RR^d}}
\newcommand{\Rk}{\ensuremath{\RR^k}}

%positive integers
\newcommand{\Zplus}{\ensuremath{\ZZ^+}}
%positive reals
\newcommand{\Rplus}{\ensuremath{\RR^+}}

% set notation
% \newcommand{\set}[1]{\ensuremath{{\left\{#1\right\}}}}
\newcommand{\set}[1]{\ensuremath{{\{#1\}}}}
\newcommand{\tuple}[1]{\ensuremath{{(#1)}}}
\newcommand{\tupleLarge}[1]{\ensuremath{{\left(#1\right)}}}
% \newcommand{\argmin}[1]{\ensuremath{\operatorname*{arg\;min}_{#1}}}
% \newcommand{\argmax}[1]{\ensuremath{\operatorname*{arg\;max}_{#1}}}
\DeclareMathOperator*{\argmin}{argmin}
\DeclareMathOperator*{\argmax}{argmax}

%Binary vector operators
\newcommand{\InnerProduct}[2]{\left\langle #1,#2 \right\rangle}
\newcommand{\CrossProduct}[2]{#1\times#2}

% Norms
% \newcommand{\norm}[1]{{{\left\|#1\right\|}}}
\newcommand{\sign}[1]{{\mathrm{sign}\left(#1\right)}}
\newcommand{\ellTwo}{\ell_2}
\newcommand{\ellTwoNorm}[1]{\norm{#1}_{\ellTwo}}
\newcommand{\ellOne}{\ell_1}
\newcommand{\ellOneNorm}[1]{\norm{#1}_{\ellOne}}

%def equiv sign
\newcommand{\defeq}{\ensuremath{\equiv}}

\newcommand{\MATRIX}[2][cccccccccccccccccccc]{\left[
 \begin{array}{#1}
 #2
 \end{array}
\right]}

% In python: print_iterable(['\\bmdefine\\b{0}'.format(x)+'{'+x+'}' for x in 
% string.ascii_letters])

% VECTORS
\bmdefine\ba{\mathrm{a}}
\bmdefine\bb{\mathrm{b}}
\bmdefine\bc{\mathrm{c}}
\bmdefine\bd{\mathrm{d}}
\bmdefine\be{\mathrm{e}}
% \bmdefine\bf{\mathrm{f}}  # Clashes with a standard command
\bmdefine\boldf{\mathrm{f}}
\bmdefine\bg{\mathrm{g}}
\bmdefine\bh{\mathrm{h}}
\bmdefine\bi{\mathrm{i}}
\bmdefine\bj{\mathrm{j}}
\bmdefine\bk{\mathrm{k}}
\bmdefine\bl{\mathrm{l}}
\bmdefine\bm{\mathrm{m}}
\bmdefine\bn{\mathrm{n}}
\bmdefine\bo{\mathrm{o}}
\bmdefine\bp{\mathrm{p}}
\bmdefine\bq{\mathrm{q}}
\bmdefine\br{\mathrm{r}}
\bmdefine\bs{\mathrm{s}}
\bmdefine\bt{\mathrm{t}}
\bmdefine\bu{\mathrm{u}}
\bmdefine\bv{\mathrm{v}}
\bmdefine\bw{\mathrm{w}}
\bmdefine\bx{\mathrm{x}}
\bmdefine\by{\mathrm{y}}
\bmdefine\bz{\mathrm{z}}

\bmdefine\balpha{\alpha}
\bmdefine\bbeta{\beta}
\bmdefine\bgamma{\gamma}
\bmdefine\bdelta{\delta}
\bmdefine\blambda{\lambda}
\bmdefine\btheta{\theta}
\bmdefine\bphi{\phi}
\bmdefine\bvarphi{\varphi}
\bmdefine\bxi{\xi}
\bmdefine\bzeta{\zeta}
\bmdefine\boldeta{\eta}
\bmdefine\bpi{\pi}
\bmdefine\bnu{\nu}
\bmdefine\bmu{\mu}
\bmdefine\brho{\rho}
\bmdefine\bomega{\omega}
\bmdefine\bOmega{\ensuremath{\Omega}}
\bmdefine\bvarepsilon{\varepsilon}
\bmdefine\bDelta{\ensuremath{\Delta}}
\bmdefine\bTheta{\ensuremath{\Theta}}
\bmdefine\bSigma{\ensuremath{\Sigma}}
\bmdefine\bPsi{\ensuremath{\Psi}}
\bmdefine\bLambda{\ensuremath{\Lambda}}
\bmdefine\bzero{0}
\bmdefine\bone{1}
\bmdefine\binfty{\infty}

\newcommand{\indicator}{\mathbbm{1}}


% In python 
% print_iterable(['\\newcommand{\\'+'{0}cal'.format(x)+'}{\\mathcal{'+x+'}}' 
% for x in string.ascii_uppercase])
\newcommand{\Acal}{\mathcal{A}}
\newcommand{\Bcal}{\mathcal{B}}
\newcommand{\Ccal}{\mathcal{C}}
\newcommand{\Dcal}{\mathcal{D}}
\newcommand{\Ecal}{\mathcal{E}}
\newcommand{\Fcal}{\mathcal{F}}
\newcommand{\Gcal}{\mathcal{G}}
\newcommand{\Hcal}{\mathcal{H}}
\newcommand{\Ical}{\mathcal{I}}
\newcommand{\Jcal}{\mathcal{J}}
\newcommand{\Kcal}{\mathcal{K}}
\newcommand{\Lcal}{\mathcal{L}}
\newcommand{\Mcal}{\mathcal{M}}
\newcommand{\Ncal}{\mathcal{N}}
\newcommand{\Ocal}{\mathcal{O}}
\newcommand{\Pcal}{\mathcal{P}}
\newcommand{\Qcal}{\mathcal{Q}}
\newcommand{\Rcal}{\mathcal{R}}
\newcommand{\Scal}{\mathcal{S}}
\newcommand{\Tcal}{\mathcal{T}}
\newcommand{\Ucal}{\mathcal{U}}
\newcommand{\Vcal}{\mathcal{V}}
\newcommand{\Wcal}{\mathcal{W}}
\newcommand{\Xcal}{\mathcal{X}}
\newcommand{\Ycal}{\mathcal{Y}}
\newcommand{\Zcal}{\mathcal{Z}}

%%%%%%%%%%%%%%%%%%%%%%%%%%%% MSc operators %%%%%%%%%%%%%%%%%%%%%%%%%%%%
\DeclareMathOperator{\E}{\mathbb{E}} %used in gaussian processes
\newcommand{\Seiz}{\ensuremath{\mathbb{S}}}
\newcommand{\EEG}{\ensuremath{\mathbb{E}}}
\newcommand{\Seizt}{\ensuremath{\mathbb{S}_t}}
\newcommand{\EEGt}{\ensuremath{\mathbb{E}_t}}

%%%%%%%%%%%%%%%%%%%%%%%%%%%% Add to PDF to TOC %%%%%%%%%%%%%%%%%%%%%%%%%%%%
% https://tex.stackexchange.com/a/88101
%   
%---------------------------- begin macro for including a PDF document
% includepdf syntax:
%     addtotoc={⟨page number⟩,⟨section⟩, ⟨level⟩,⟨heading⟩,⟨label⟩}
%     addtolist={⟨page number⟩,⟨type⟩,⟨heading⟩,⟨label⟩}
%   \IncludeMyPDF
%   {1} %  page number to be included
%   {0.9} % scale
%   {true} %   landscape = true or false
%   {false} %  turn = true or false
%   {subsection,2} % level in TOC: section, subsection, subsubsection + level 1,2,3
%   {TitleTOC} %  heading for TOC / list 
%   {Label} %   label: label-toc-#7, label-list-#7, #7-target for hyperlinks
%   {table} %   addtolist = table or figure
%   {mindmaps.pdf} %  file

\newcommand{\IncludeMyPDF}[9]{%
\newpage\hypertarget{#7-target}
{\includepdf[pages={#1},nup=1x1,
    scale=#2,landscape=#3,turn=#4,
    pagecommand={\thispagestyle{empty}},
    addtotoc={#1,#5,#6,label-toc-#7},
    addtolist={#1,#8,#6,label-list-#7}]
{#9}}}

\newcommand{\IncludeMyPDFinReverse}[9]{%
\newpage\hypertarget{#7-target}
{\includepdf[pages={#1},nup=1x1,
    scale=#2,landscape=#3,turn=#4,
    pagecommand={\thispagestyle{empty}},
    addtotoc={#1,#5,#6,label-toc-#7},
    addtolist={#1,#8,#6,label-list-#7},
    pages=last-1]
{#9}}}
