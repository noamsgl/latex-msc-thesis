% #################### ENCODING ####################
\usepackage[utf8x]{inputenc} %allow characters beyond ASCII, such as naïve and angled brackets
\usepackage[T1]{fontenc} %change font encoding to T1
%\usepackage[USenglish]{babel} %We don't need babel when we have english only

% textcomp package and marvosym package for additional characters
\usepackage{textcomp,marvosym}

% #################### FONT ####################
\usepackage[lining]{ebgaramond} %artistic font, uses more readable numbers
%took this from the Harvard template: https://github.com/suchow/Dissertate

\usepackage[normalem]{ulem}
% font tricks, ie, strikeout \sout{some text}
% see http://texdoc.net/texmf-dist/doc/generic/ulem/ulem.pdf

\usepackage[export]{adjustbox} %used in covers for underlining signatures

\usepackage{siunitx}
% scientific notation,
% https://tex.stackexchange.com/a/269845

% #################### COMMENTS ####################
\usepackage{comment} %multi line comment, a comment environment

% #################### DATES ####################
\usepackage[en-US]{datetime2}

% #################### ALGORITHM ####################
\usepackage{algpseudocode}
\usepackage{algorithm}

% #################### GLOBAL THESIS CONFIGURATIONS ####################
%%%%%%%%%%%%%%%%%%%%%%%%%%%%%%%%%%%%%%%%%%%%%%%%%%%%%%%%%%%%%%%%%%%%%%%%%%%%%%%%
%% DRAFT conditional
%%
\usepackage{ifdraft}
\newif\ifDRAFT

% on production mode: 
%  NO lineno, todos, links colors for debugging
%  Hi-res images

%sync with overleaf/documentclass[draft] mode
\ifdraft{\DRAFTtrue}{}

% \DRAFTtrue %FUTURE: remove this
%forcing draft for reading version to committee

%%%%%%%%%%%%%%%%%%%%%%%%%%%%%%%%%%%%%%%%%%%%%%%%
%Title and names for the article
\newcommand{\thesistitle}{\textbf{Computational Approaches to Seizure Forecasting}}
\newcommand{\thesistitlehe}{\textbf{כותרת בעברית של התזה}}

\newcommand{\thesisauthorname}{\textbf{Noam Siegel}}
\newcommand{\thesisauthornamehe}{\textbf{שם המחבר}}

\newcommand{\thesissupervisername}{\textbf{Prof. Oren Shriki and Dr. David Tolpin}}
\newcommand{\thesissupervisernamehe}{\textbf{שם המנחה}}

\newcommand{\thesismonth}{\textbf{January}}
\newcommand{\thesismonthhe}{\textbf{חודש}}
\newcommand{\thesisyear}{\textbf{2023}}
%%%%%%%%%%%%%%%%%%%%%%%%%%%%%%%%%%%%%%%%%%%%%%%%

%%%%%%%%%%%%%%%%%%%%%%%%%%%%%%%%%%%%%%%%%%%%%%%%%%%%%%%%%%%%%%%%%%%%%%%%%%%%%%%%
%% Thesis metadata:
%%
\def\thesisTitle{Computational Approaches to Seizure Forecasting}
\def\thesisAuthor{Noam Siegel}
\DTMsavedate{submissiondate}{2023-01-16}   % Submission date
% \DTMsavenow{submissiondate}   % Submission date
\def\thesisKeywords{Computer Science - Machine Learning;Statistics - Machine Learning;Bayesian;Epilepsy;Seizure Prediction;Weak Supervision}

%%%%%%%%%%%%%%%%%%%%%%%%%%%%%%%%%%%%%%%%%%%%%%%%%%%%%%%%%%%%%%%%%%%%%%%%%%%%%%%%
%% conditionals that affect performance
%%
\newif\ifFULL
\FULLtrue %Renders everything (prologue, epilogue, all chapters, indexes) - longer compilation

\newif\ifFINAL
\FINALtrue %Does not render design-time stuff, e.g. tcolorbox

\newif\ifCODE
\CODEfalse % minted and listing in homer chapter

\newif\ifHEBCOVER
\HEBCOVERtrue %Renders the hebrew cover at the end of the phd

\newif\ifAPPENDIX
% \APPENDIXtrue %Renders appending (only if full is true)

\newif\ifTODOS
% \ifFINAL
% \else
% \fi
\TODOStrue % shows todo notes


\newif\ifLISTFIGS
\ifFINAL
\else
    % \LISTFIGStrue %shows a list of figures in the end, easier debugging
\fi

\newif\ifLINENO
% \ifFINAL
% \else
% \fi
% \LINENOtrue %renders line numbers


\newif\ifBIB
\BIBtrue %compiles the bibliography

\newif\ifBADBOXES 
\ifFINAL
\else
    % \BADBOXEStrue %visualize bad-boxes
\fi



% #################### SI Units ####################
% %https://tex.stackexchange.com/a/2254/199031
% \usepackage{siunitx}
% \sisetup{load-configurations = abbreviations, range-phrase = --, range-units=single}

% #################### COLORS ####################
\usepackage[pdftex,dvipsnames]{xcolor}  % Coloured text etc.

% #################### WATERMARK ####################
\ifDRAFT 
    \usepackage{draftwatermark}
    
    \SetWatermarkText{\textbf{D R A F T}  \normalsize\today} 
    \SetWatermarkColor[gray]{0.7}
    \SetWatermarkFontSize{1cm}
    \SetWatermarkAngle{90}
    \SetWatermarkHorCenter{.7cm}
\fi

% #################### INDEXES ####################
\usepackage{etoolbox} %ifstrempty
%% acronyms in index
\def\ixML{Machine learning (ML)}
\def\ixSVM{Support Vector Machine (SVM)} 
\def\ixLDA{Linear Discriminant Analysis (LDA)}
\def\ixPCA{Principal Component Analysis (PCA)}
\def\ixRF{Random Forest (RF)}
\def\ixET{Extra Trees (ET)}
\def\ixGBDT{Gradient Boosting Decision Tree (GBDT)}

\def\ixHP{Hyper-Parameters (HP)}
\def\ixHPO{Hyper-parameter optimization (HPO)}
\def\ixANN{Artificial Neural Networks (ANN)}
\def\ixNAS{Neural Architecture Search (NAS)}
\def\ixAutoML{Automated Machine Learning (AutoML)}
\def\ixCMAES{Covariance Matrix Adaptation - Evolution Strategy (CMA-ES)}
\def\ixGP{Gaussian Process (GP)}
\def\ixPBT{Population Based Training (PBT)}

\def\ixBO{Bayesian Optimization (BO)}
\def\ixSMBO{Sequential Model-Based Optimization (SMBO)}
\def\ixTPE{Tree Parzen Estimator (TPE)}
\def\ixEI{Expected Improvement (EI)}

\def\ixXAI{eXplainable Artificial Intelligence (XAI)}
\def\ixCBR{Case-Based Reasoning (CBR)}
\def\ixEDA{Exploratory Data Analysis (EDA)}
\def\ixPDP{Partial Dependence Plots (PDPs)}
\def\ixICE{Individual Conditional Expectation (ICE)}
\def\ixALE{Accumulated Local Effects (ALE)}
\def\ixLIME{Local Interpretable Model-agnostic Explanations (LIME)}
\def\ixSHAP{SHapley Additive exPlanations (SHAP)}
\def\ixSCM{Structural Causal Models (SCM)}

\def\ixEEG{Electroencephalography (EEG)}


\def\ixSPD{Seizure Predictive Devices (SPDs)}

\def\ixEIB{Excitation-Inhibition Balance (EIB)}
\def\ixICA{Independent Component Analysis (ICA)}
\def\ixIC{Independent Component (IC)}

\def\ixSTDP{Spike Timing-Dependent Plasticity (STDP)}
\def\ixSRM{Spike Response Model (SRM)}
\def\ixPSP{Post Synaptic Potential (PSP)}
\def\ixIF{Integrate-and-Fire neuron (IF)}
\def\ixKL{Kullback Leibler (KL)}
\def\ixLSM{Liquid State Machine (LSM)}

%% index making logic
\ifFULL
    \usepackage{imakeidx}
    \makeindex[columns=2, title=Alphabetical Index, intoc, options= -s 00Preamble/index.ist]

    %print index parameter, https://tex.stackexchange.com/a/54519/199031
    \let\oldindex\index
    % \ifDRAFT
        %color indexes
        \renewcommand*{\index}[2][]{%
          \ifstrempty{#1}{%
            \textcolor{DarkOrchid!100}{\textsf{#2}}\oldindex{#2}%
          }{%
            \textcolor{DarkOrchid!100}{\textsf{#2}}\oldindex{#1}%
          }%
        }%
    % \else
    %     \renewcommand*{\index}[2][]{%
    %       \ifstrempty{#2}{%
    %         \oldindex{#1}%
    %       }{%
    %         #2\oldindex{#1}%
    %       }%
    %     }
    % \fi
\else %color only
    \renewcommand*{\index}[2][]{%
          \ifstrempty{#2}{%
            \textcolor{DarkOrchid!100}{\textsf{#1}}%
          }{%
            \textcolor{DarkOrchid!100}{\textsf{#2}}%
          }%
        }%
\fi

% #################### APPENDIX ####################
\usepackage[page,title,titletoc,header]{appendix}
%TODO:
%\renewcommand{\appendixpagename}{\appendixname}
%\renewcommand{\appendixtocname}{\appendixname}
%\noappendicestocpagenum

% #################### MATH ####################
% \usepackage{mathpazo} % Use the Palatino font by default
\usepackage{amsmath,amssymb,amsfonts}
%amssymb: overleaf warn when using \mathbb
%amsfonts: convension to import the 3 of them

% Definitions environment
\usepackage{amsthm}   %error when used with regexpatch

\usepackage{mathtools} %use of middle

\usepackage{physics} %used in brackets

%better math font
% \usepackage[osf,onlytext]{MinionPro}% use osf in text, lining figures in math

\usepackage[cochineal]{newtxmath}
% other options as of March 2022
% libertine
% libertinus
% etbb --> ETbb
% ebgaramond
% MinionPro
% minion --> MinionPro
% cochineal
% garamondx
% baskervillef
% baskerville --> baskervillef
% Baskerville --> baskervillef
% BaskervilleF --> baskervillef
% baskervaldx
% Baskervaldx --> baskervaldx
% erewhon
% Erewhon --> erewhon
% XCharter
% xcharter --> XCharter
% stickstoo --> stickstootext
% Stickstoo --> stickstootext
% stix2 --> stickstootext
% scholax
% nc --> scholax
% scholaxf
% ncf --> scholaxf

% \usepackage{commath} % extra functionalities for differentials, integrals
% etc (\od, \dif, etc) #FUTURE: ?

%nice fractions
\usepackage{nicefrac}

% #################### LISTS ####################
\usepackage{enumitem} %enumeration styling
%https://www.latex-tutorial.com/tutorials/lists/

 % #################### TABULAR ####################
 \usepackage{tabu} %better tabular formatting, ex. https://tex.stackexchange.com/a/50337/199031
 \usepackage{multirow} %multi rows in tables

%  \usepackage{colortbl} %coloring of table rows


 % #################### LINE NUMBERS ####################
\ifLINENO
    %Line numbers: see https://texblog.org/2012/02/08/adding-line-numbers-to-documents/
    \usepackage[left,running]{lineno}
    \renewcommand\linenumberfont{\ttfamily\bfseries\scriptsize}
    
    %lineno package loading order: https://tex.stackexchange.com/a/447159/199031
    
    %Patch for equations:
    % https://tex.stackexchange.com/a/443201/199031
    % [mathlines] option is full of bugs
    \usepackage{etoolbox} %% <- for \pretocmd and \apptocmd
    \makeatletter %% <- make @ usable in macro names
    \newcommand*\linenomathpatch{\@ifstar{\linenomathpatch@AMS}{\linenomathpatch@}}
    \newcommand*\linenomathpatch@[1]{
      \expandafter\pretocmd\csname #1\endcsname {\linenomathWithnumbers}{}{}
      \expandafter\pretocmd\csname #1*\endcsname{\linenomathWithnumbers}{}{}
      \expandafter\apptocmd\csname end#1\endcsname {\endlinenomath}{}{}
      \expandafter\apptocmd\csname end#1*\endcsname{\endlinenomath}{}{}
    }
    \newcommand*\linenomathpatch@AMS[1]{
      \expandafter\pretocmd\csname #1\endcsname {\linenomathWithnumbersAMS}{}{}
      \expandafter\pretocmd\csname #1*\endcsname{\linenomathWithnumbersAMS}{}{}
      \expandafter\apptocmd\csname end#1\endcsname {\endlinenomath}{}{}
      \expandafter\apptocmd\csname end#1*\endcsname{\endlinenomath}{}{}
    }
    \let\linenomathWithnumbersAMS\linenomathWithnumbers
    \patchcmd\linenomathWithnumbersAMS{\advance\postdisplaypenalty\linenopenalty}{}{}{}
    \makeatother %% revert @
    
    \linenomathpatch{equation}
    \linenomathpatch*{align}
    \linenomathpatch*{subequations}
\fi

% #################### FIGURES ####################
% \graphicspath{{./}} %We've decides to use full path from graphics

%In draft mode, display low-res figures
%https://tex.stackexchange.com/a/74901/199031

\ifDRAFT
    \DeclareGraphicsExtensions{.png} %use low-res version. - ~10 sec post compilation time
\else
    \DeclareGraphicsExtensions{.pdf} %use high-res version
\fi

% used for trees in 2HOMER chapter
\usepackage{tikz}
\usetikzlibrary{bayesnet}
% \usetikzlibrary{trees}

\usepackage{caption}
% docs: http://ftp.isu.edu.tw/pub/Unix/CTAN/macros/latex/contrib/caption/caption.pdf
% for figures: caption label is bold, the caption text normal.
% justification is RaggedRight (i.e. left aligned)
% singlelinecheck=off means that the justification setting is used even when the caption is only a single line long.
% indention: used to alight the text under the label
\newlength{\floatwidth}
\setlength{\floatwidth}{0.85\textwidth}
\captionsetup[figure]{font={footnotesize,sf}, labelfont=bf,justification=justified,singlelinecheck=off,labelsep=space,format=plain,width=\floatwidth}

% Aviv: I do not like the use of subcaptions, and these lines threw a warning about unused caption[sub]. 
\usepackage{subcaption}
% \captionsetup[sub]{indention=1pt}

%tables
\usepackage{tabularx} %more advanced package for tables, allows sizing, see https://www.overleaf.com/learn/latex/tables

\usepackage{booktabs} %allows export from pandas: \toprule, \midrule, \bottomrule

\captionsetup[table]{font={footnotesize,sf}, labelfont=bf,singlelinecheck=off,justification=justified,labelsep=space,format=plain,width=\floatwidth}

\ifCODE
    %source code
    \captionsetup[listing]{font={footnotesize,sf}, labelfont=bf,singlelinecheck=off,justification=justified,labelsep=space,format=plain,width=\floatwidth}
\fi

%we use custom command `Caption`!
% NOTE: don't put \index[] inside caption #1, we don't know how to handle it currently
\newcommand*{\Caption}[2]{
    \caption[#1]{
        \textbar\, 
        \textbf{#1.} 
        #2}
}


% \renewcommand\thesubfigure{\Alph{subfigure}} %subfigures will be numbered (A), (B) instead of (a),(b)

% list of figures customization
\ifLISTFIGS
    \usepackage{tocloft}
    \renewcommand*{\cftfigname}{\figurename\space}
    \renewcommand*{\cftfigaftersnum}{\textbar~}
    \renewcommand{\cftfigpresnum}{\cftfigname}
    \setlength{\cftfigindent}{0pt}
    \setlength{\cftfignumwidth}{\widthof{\cftfigname 0.00\textbar~}}
\fi

\ifCODE
% #################### CODE FORMATTING ####################
    %FUTURE: using matlab code: https://tex.stackexchange.com/questions/75116/what-can-i-use-to-typeset-matlab-code-in-my-document

    % source code formatting
    \usepackage[newfloat, %allows caption to be on top
    chapter, %number by chapter
    final %color the code even if on draft mode
    ]{minted} %color the code even if on draft mode
    \usemintedstyle{perldoc}
    % see https://pygments.org/demo/#try for demo of styles
\fi

% #################### IMPORTING PDFS ####################
\usepackage[final]{pdfpages} %import hebrew pdf later or temporary seizure prediction paper pdf

% #################### Quote Blocks ####################
% https://tex.stackexchange.com/a/325698/199031
% \usepackage{etoolbox}
\usepackage{setspace} % for \onehalfspacing and \singlespacing macros
\AtBeginEnvironment{quote}{\singlespacing\small}
% IMPORTANT: has to come before hyperref - otherwise footnotes won't work
% https://stackoverflow.com/q/52954054

% #################### REFERENCING ####################
\ifBIB
    
    \ifPRINT
        \newcommand{\linkcolor}{black}
        \newcommand{\urlcolor}{black}
        \newcommand{\citecolor}{black}
    \else
        \newcommand{\linkcolor}{violet}    % Color normal internal links (e.g. toc, eqrefs)
        \newcommand{\urlcolor}{blue!75!black}       % Color for linked urls
        \newcommand{\citecolor}{blue!50!black}      % Color for bibliographical citations in text
    \fi

    \usepackage{varioref}
    % \usepackage[bookmarks=true]{hyperref}

    %suggested by aviv, adapts automatic zoom to fit page
    \usepackage[colorlinks,linktocpage,linktoc=all,pdfstartview=]{hyperref}

    \hypersetup{ %some pdf metadata
        draft=false, %discard draft mode (render links anyway)
        pdfauthor = {\thesisAuthor},
        pdftitle = {\thesisTitle},
        pdfsubject={\thesisTitle},
        pdfkeywords={\thesisKeywords},
        pdfproducer={pdfLaTeX},
        pdfcreator={pdfLaTeX},
        bookmarksopen=false, %starts document with all subtrees expanded?
        bookmarksnumbered=true, %Include section numbers in bookmarks
        %pdfpagemode = FullScreen, %open in fullscreen (e.g. not thumbnail)
        % Kile will display a dialog asking whether to allow fullscreen, than if we choose "Yes",
        %   is will crush!!!
        %plainpages=true, % FUTURE: do we need this?
        linktoc=all, %make text (section), page number (page), both (all) or nothing (none)
        colorlinks = true, %color or boxes (Better for debug)
        linkcolor=\linkcolor, %Color for normal internal links (e.g. toc, eqrefs), %FUTURE: black
        citecolor=\citecolor, %Color for bibliographical citations in text
        urlcolor = \urlcolor, %Color for linked urls
    }
    % \usepackage[noabbrev]{cleveref} %noabbrev means display `equation` instead of `eq.`
    % \usepackage[capitalize]{cleveref} 
    \usepackage{cleveref}
    \crefname{figure}{}{}

    % omit () from equations when referencing, https://tex.stackexchange.com/a/373064/199031
    \creflabelformat{equation}{#2\textup{#1}#3}
    
    \let\oldcref\cref %it can be used explicitly
    \renewcommand{\cref}[1]{\oldcref{#1}} %allows us to customize cref
    
%     \AtBeginEnvironment{appendices}{\crefalias{section}{Appendix}}  % Set appendix reference correctly
    
    % \usepackage[open,openlevel=0]{bookmark} %custom pdf bookmarks (e.g. Declaration)
    \usepackage{bookmark}
\else
    \renewcommand{\cite}[1]{[\textbf{C?}]} %default behavior: placeholder
    \renewcommand{\citet}[1]{[\textbf{Ct?}]} %default behavior: placeholder
    \renewcommand{\ref}[1]{[\textbf{R?}]} %default behavior: placeholder
    \renewcommand{\cref}[1]{[\textbf{cR?}]} %default behavior: placeholder
    \renewcommand{\nameref}[1]{[\textbf{nR?}]} %default behavior: placeholder
    \renewcommand{\oldcref}[1]{[\textbf{oR?}]} %default behavior: placeholder
\fi

% Theorem environments and such
% Must be defined AFTER clerveref is loaded
\newtheorem{theorem}{Theorem}
\newtheorem{corollary}[theorem]{Corollary}
\newtheorem{lemma}[theorem]{Lemma}
\theoremstyle{definition}
\newtheorem{definition}{Definition}[chapter]
\theoremstyle{remark}
\newtheorem{remark}[theorem]{Remark}
% amsthm replacements (https://tex.stackexchange.com/questions/562490/how-to-customize-newtheorem)
% (I leave those here just in case)
% \makeatletter
% \@ifdefinable\Olddefinition{\let\Olddefinition=\definition}%
% \renewcommand\definition{\@ifnextchar[{\definitionopt}{\Olddefinition\normalfont}}%
% \newcommand\definitionopt[1][]{\Olddefinition[{#1}]\normalfont}%
% \makeatother
% \makeatletter
% \@ifdefinable\Oldremark{\let\Oldremark=\remark}%
% \renewcommand\remark{\@ifnextchar[{\remarkopt}{\Oldremark\normalfont}}%
% \newcommand\remarkopt[1][]{\Oldremark[{#1}]\normalfont}%
% \makeatother
% \newenvironment{proof}{\paragraph{\normalfont\fontfamily{cmr}\selectfont\textit{Proof.}\hspace{-0.5em}}}{\hfill$\square$}

%hyperref transforms every \cite to hyperlink

%\usepackage[authoryear,square,longnamesfirst,comma,sort]{natbib}   % Shay's formatting
\usepackage[numbers,square,comma,sort&compress,elide]{natbib}
%nice cheatsheet: http://merkel.texture.rocks/Latex/natbib.php
% \bibliographystyle{plainnat}

% % In-text full references
% % https://tex.stackexchange.com/questions/93845/suppress-printing-of-only-bibentry-references-with-natbib
% \usepackage{multibib}
% \newcites{all}{Bibliography}
% \usepackage{bibentry}
% \nobibliography*

% abort line breaks in multiple citations

% this is just for convenience, if it causes problems we can unuse it
\renewcommand{\cite}{\citep}
% \renewcommand{\cite}{\citepall}

%breaks long links, not causing hbox warns
%especially useful when \bibliographystyle{apalike} %First author, year
% \bibliographystyle{apalike}
%FUTURE: Will we need it when switching to numerics (\bibliographystyle{unsrt})?
% \usepackage{breakcites}

% DOI hyperlinks
\usepackage{doi}

% #################### PHANTOM CHAPTER ####################
% pantom chapter so that hebrew chapter appears in toc without number
% https://tex.stackexchange.com/a/314780
\providecommand{\phantomsection}{}

\makeatletter

\let\l@chapternonum\l@section %chapter
\newcommand{\@chapternonum}[2][]{\phantomsection\addcontentsline{toc}{chapter}{#1}\edef\@currentlabel{#1}}%
\newcounter{chapternonum}
\renewcommand{\thechapternonum}{} %[1]{\chapter{#1}}
\makeatother

%%%%%%%%%%%%%%%%%%%%%%%%%%%%%%%%%%%%%%%%%%%%%%%%%%%%%%%%%%%%%%%%%%%%%%%%%%%%%%%%
% PREFACE:

% ===== Define abstract environment =====
\newcommand{\prefacename}{Preface}
\newenvironment{preface}{
    \vspace*{\stretch{2}}
    {\noindent \bfseries \Huge \prefacename}
    \begin{center}
        \phantomsection \addcontentsline{toc}{chapter}{\prefacename} % enable this if you want to put the preface in the table of contents
        \thispagestyle{plain}
    \end{center}%
}
{\vspace*{\stretch{5}}}

%   signature line
% https://tex.stackexchange.com/a/48156
% \newcommand*{\SignatureAndDate}[1]{%
%     \par\noindent\makebox[4cm]{\hrulefill} \hfill\makebox[4cm]{\hrulefill}%
%     \par\noindent\makebox[4cm][l]{#1}      \hfill\makebox[4cm][l]{Date:\\dute}%
% }%

\newcommand*{\SignatureAndDate}[1]{%
    \par\noindent\makebox[5.5cm]{\hrulefill} \hfill\makebox[2.5cm]{\hrulefill}%
    \par\noindent\parbox{5.5cm}{#1}      \hfill\makebox[2.5cm][r]{Date}%
}%
% #################### PAGE LAYOUT ####################

\usepackage{anysize} % Customize margins
\ifTODOS
    %little more on the right for todonotes
    \marginsize{2cm}{2.2cm}{2cm}{2cm} % Left,right, up, down
\else
    \marginsize{2cm}{2cm}{2cm}{2cm}
\fi

\renewcommand{\baselinestretch}{1} %spacing between lines
\setlength{\parskip}{.5em} %skip one line between paragraphs
% \setlength{\parindent}{0em} %no indent at the beginning of a paragraph

% `fancyhdr` not compatible with KOMA-script
% see https://youtu.be/TQtrFsV3O5c
\usepackage[headsepline=true, autooneside=false]{scrlayer-scrpage}
\ifBADBOXES
    %renders a black-box suffix where a black box warning occurs
\else
    \KOMAoptions{draft=false} % disables the draft mode for scrlayer-scrpage, removed buggy rulers
\fi

%FUTURE: (maybe) before submitting PhD turn this off and see if its beneficial or not
\usepackage{microtype} %reduces the number of hyphenations, see https://tex.stackexchange.com/q/95608/199031

\usepackage{csquotes} %use enquote{} to fix for english quotes

%quote an epigraph
\usepackage{epigraph}
\renewcommand\epigraphflush{center}
\setlength{\epigraphwidth}{0.75\textwidth}
\epigraphnoindent

%dedication horizontal alignment
\usepackage{varwidth}

\interfootnotelinepenalty=10000 %Completely prevent breaking of footnotes across pages, https://tex.stackexchange.com/a/32210/199031

% #################### TODOs ####################
\usepackage{xargs} % Use more than one optional parameter in a new commands

\ifTODOS
    %\usepackage{showframe} %allows debugging the margins
    \setlength {\marginparsep } {2mm}
    \setlength {\marginparwidth }{2cm} %todonotes warn
    \usepackage[colorinlistoftodos,prependcaption,textsize=small]{todonotes}
%     \newcommandx{\OS}[2][2=fancyline]{\todo[linecolor=Gray,backgroundcolor=Gray!25,bordercolor=Gray,#2]{\textsf{\textit{OS}: #1}}}
%     \newcommandx{\SB}[2][2=fancyline]{\todo[linecolor=OliveGreen,backgroundcolor=OliveGreen!25,bordercolor=OliveGreen,#2]{\textsf{\textit{SB}: #1}}}
    % \newcommandx{\AD}[2][2=fancyline]{\todo[linecolor=Plum,backgroundcolor=Plum!25,bordercolor=Plum,#2]{\textsf{\textit{AD}: #1}}}
%     
%     \newcommandx{\SBOS}[3][2=fancyline,3=RoyalBlue]{\todo[linecolor=#3,backgroundcolor=#3!25,bordercolor=#3,#2]{\textsf{\textit{SB4OS}: #1}}}
%     
%     \newcommandx{\SBOSHIGH}[3][2=fancyline,3=Red]{\todo[linecolor=#3,backgroundcolor=#3!25,bordercolor=#3,#2]{\textsf{\textbf{\textit{SB4OS}: #1}}}}
%     
%     \newcommandx{\SBAD}[2][2=fancyline]{\todo[linecolor=Apricot,backgroundcolor=Apricot!25,bordercolor=Apricot,#2]{\textsf{\textit{SB4AD}: #1}}}

    % \newcommandx{\improvement}[2][1=]{\todo[linecolor=Plum,backgroundcolor=Plum!25,bordercolor=Plum,#1]{#2}}
    %2nd parameter: optional [fancyline, inline]
    %An example of inline comment:
    %\OS{A comment}[inline]

%     \usepackage{regexpatch}       % already used within the amsthm package, uncomment if necessary
    \makeatletter
    \xpatchcmd{\@todo}{\setkeys{todonotes}{#1}}{\setkeys{todonotes}{fancyline,#1}}{}{}
    \makeatother
    \tikzset{/tikz/notestyleraw/.append style={text=blue!10!black}}
    \newcommand{\NS}[2][]{\todo[author=NS, linecolor=Plum,backgroundcolor=Plum!25,bordercolor=Plum, #1]{#2}}
    \newcommand{\AD}[2][]{\todo[author=ad, linecolor=Yellow,backgroundcolor=Yellow!25,bordercolor=Yellow, #1]{#2}}
    \newcommand{\OS}[2][]{\todo[author=OS, linecolor=RoyalBlue,backgroundcolor=RoyalBlue!25,bordercolor=RoyalBlue, #1]{#2}}
    \newcommand{\DT}[2][]{\todo[author=DT, linecolor=OliveGreen,backgroundcolor=OliveGreen!25,bordercolor=OliveGreen, #1]{#2}}
%    \newcommand{\NS}[2][]{\todo[author=NS, linecolor=red,backgroundcolor=red!50,bordercolor=red, #1]{#2}}
%    \newcommand{\MG}[2][]{\todo[author=MG, linecolor=Rhodamine,backgroundcolor=Rhodamine!50,bordercolor=Rhodamine, #1]{#2}}
%    \newcommand{\MD}[2][]{\todo[author=MD, linecolor=Emerald,backgroundcolor=Emerald!50,bordercolor=Emerald, #1]{#2}}
%    \newcommand{\RM}[2][]{\todo[author=RM, linecolor=Gray,backgroundcolor=Gray!25,bordercolor=Gray, #1]{#2}}
    %An example of inline comment:
    %\OS[inline]{A comment}
    
    % Add progress bars
    \usepackage[width=\textwidth, heightr=1.5]{progressbar}
\else
    %ignore custom todo commands
    \newcommand{\NS}[2][]{\ignorespaces}
    \newcommand{\OS}[2][]{\ignorespaces}
    \newcommand{\DT}[2][]{\ignorespaces}
%    \newcommand{\NS}[2][]{\ignorespaces}
%    \newcommand{\MG}[2][]{\ignorespaces}
%    \newcommand{\MD}[2][]{\ignorespaces}
%    \newcommand{\RM}[2][]{\ignorespaces}
\fi

% #################### Supplementary ####################
\crefname{supfigure}{Supp. fig.}{Supp. figs.}
\Crefname{supfigure}{Supporting fig.}{Supporting figs.}

% #################### PLACEHOLDERs ####################
%\usepackage{framed} %using `oframed` environment we can add placeholders when creating the document scaffolding

\usepackage[most]{tcolorbox} %%extends `framed`, better on multi-page
% https://tex.stackexchange.com/a/184683/199031
\tcbset{colback=white}

% #################### Empty footnotes ####################
\newcommand\extrafootertext[1]{%
    \bgroup
    \renewcommand\thefootnote{\fnsymbol{footnote}}%
    \renewcommand\thempfootnote{\fnsymbol{mpfootnote}}%
    \footnotetext[0]{#1}%
    \egroup
}

\usepackage{bbm} %for indicator
\usepackage{bm} %bold math stuff

% #################### CUSTOM COMMANDS ####################
\newcommand\blankPage{
\newpage
\mbox{}
\newpage
}

\newcommand\blankPageWithoutFooter{
\newpage
\mbox{}
\thispagestyle{empty} %no pagemark footer here
\newpage
}

\newcommand\blankPageWithoutHeader{
\newpage
\mbox{}
\thispagestyle{plain} %no header
\newpage
}

\newcommand\setPageStyle{
\pagestyle{scrheadings} %custom header/footer

\ifPRINT
    \ohead{\leftmark}
    \chead{}
    \ihead{\rightmark}
    \ofoot[\pagemark]{\pagemark}
    \cfoot[]{}
    \ifoot[]{}
\else
    \lohead{\leftmark}
    \cohead{}
    \rohead{\rightmark}
    \lehead{\leftmark}
    \cehead{}
    \rehead{\rightmark}
    \refoot[\pagemark]{\pagemark}
    \cefoot[]{}
    \lefoot[]{}
    \rofoot[\pagemark]{\pagemark}
    \cofoot[]{}
    \lofoot[]{}
\fi
\automark[section]{chapter}
}

% #################### abbreviations ####################

 %--------------------------------------------
% Add a period to the end of an abbreviation unless there's one
% already, then \xspace.
% \DeclareRobustCommand\onedot{\futurelet\@let@token\@onedot}
% \def\@onedot{\ifx\@let@token.\else.\null\fi\xspace}
% \def\@onedot{\xspace}

\def\eg{\emph{e.g.}~}  %for example, such as
\def\ie{\emph{i.e.,~}} %in other words
% \def\cf{\emph{cf.}~} \def\Cf{\emph{Cf.}~}
\def\etc{\emph{etc.}~} \def\vs{\emph{vs.}~}
\def\wrt{w.r.t.~} 
% \def\wrt{with respect to }  % The extra space at the end means \wrt blah would 
% not end up as ``with respect toblah''. 
\def\aka{a.k.a.~} 

% \def\dof{d.o.f.~}
% \def\etal{\emph{et al.}~}
\def\etal{\emph{et al.}~}
% \def\etal{\emph{et al.}~}
%
\def\iid{\emph{i.i.d.}~}
% \def\pdf{\emph{p.d.f.}~}

\def\rhs{r.h.s.~}
\def\lhs{l.h.s.~}

\newcommand{\supth}{\ensuremath{^{\text{th}}}} %\th was already occupied, usage: e.g. for k^th

%--------------------------------------------------
\def\naive{naïve~}
\def\Naive{Naïve~}
% TODO: add Matern

% \newcommand{\FIG}{Fig.~}
% \newcommand{\FIGS}{Figs.~}
% \newcommand{\EQN}{Eq.~}
% % \newcommand{\EQN}{Equation }
% \newcommand{\EQNS}{Eqs.~}
% \newcommand{\EQNS}{Equations }

%\newcommand{\SEC}{Sec.~}.  Use \autoref or ~\autoref instead.

% #################### Math convenience ####################

%https://tex.stackexchange.com/a/5255/199031
% \DeclareMathOperator*{\argmax}{arg\,max}
% \DeclareMathOperator*{\argmin}{arg\,min}

%About \ensuremath{#1}:
% The aim of it to allow #1 to be used in math mode and outside.
% yet, as we can see in this link, there sometimes may be problems with spacing when outside of mathmode.
% My take is use it only when you have a single letter, \ie \newcommand{\ZZ}{\ensuremath{\mathbb{Z}}}
% ans sparingly in other cases
%https://tex.stackexchange.com/questions/34830/when-not-to-use-ensuremath-for-math-macro

% probability
\newcommand{\given}{\ensuremath{\;\middle\vert\;}} %has to be inside \left \right
\newcommand{\prob}{\mathbb{P}}
% integers
\newcommand{\ZZ}{\ensuremath{\mathbb{Z}}}
\newcommand{\RR}{\ensuremath{\mathbb{R}}}
% \newcommand{\Rtwo}{\ensuremath{\RR^2}}
% \newcommand{\Rthree}{\ensuremath{\RR^3}}
% \newcommand{\Rsix}{\ensuremath{\RR^6}}
\newcommand{\Rn}{\ensuremath{\RR^n}}
\newcommand{\Rd}{\ensuremath{\RR^d}}
\newcommand{\Rk}{\ensuremath{\RR^k}}

%positive integers
\newcommand{\Zplus}{\ensuremath{\ZZ^+}}
%positive reals
\newcommand{\Rplus}{\ensuremath{\RR^+}}

% set notation
% \newcommand{\set}[1]{\ensuremath{{\left\{#1\right\}}}}
\newcommand{\set}[1]{\ensuremath{{\{#1\}}}}
\newcommand{\tuple}[1]{\ensuremath{{(#1)}}}
\newcommand{\tupleLarge}[1]{\ensuremath{{\left(#1\right)}}}
% \newcommand{\argmin}[1]{\ensuremath{\operatorname*{arg\;min}_{#1}}}
% \newcommand{\argmax}[1]{\ensuremath{\operatorname*{arg\;max}_{#1}}}
\DeclareMathOperator*{\argmin}{argmin}
\DeclareMathOperator*{\argmax}{argmax}

%Binary vector operators
\newcommand{\InnerProduct}[2]{\left\langle #1,#2 \right\rangle}
\newcommand{\CrossProduct}[2]{#1\times#2}

% Norms
% \newcommand{\norm}[1]{{{\left\|#1\right\|}}}
\newcommand{\sign}[1]{{\mathrm{sign}\left(#1\right)}}
\newcommand{\ellTwo}{\ell_2}
\newcommand{\ellTwoNorm}[1]{\norm{#1}_{\ellTwo}}
\newcommand{\ellOne}{\ell_1}
\newcommand{\ellOneNorm}[1]{\norm{#1}_{\ellOne}}

%def equiv sign
\newcommand{\defeq}{\ensuremath{\equiv}}

\newcommand{\MATRIX}[2][cccccccccccccccccccc]{\left[
 \begin{array}{#1}
 #2
 \end{array}
\right]}

% In python: print_iterable(['\\bmdefine\\b{0}'.format(x)+'{'+x+'}' for x in 
% string.ascii_letters])

% VECTORS
\bmdefine\ba{\mathrm{a}}
\bmdefine\bb{\mathrm{b}}
\bmdefine\bc{\mathrm{c}}
\bmdefine\bd{\mathrm{d}}
\bmdefine\be{\mathrm{e}}
% \bmdefine\bf{\mathrm{f}}  # Clashes with a standard command
\bmdefine\boldf{\mathrm{f}}
\bmdefine\bg{\mathrm{g}}
\bmdefine\bh{\mathrm{h}}
\bmdefine\bi{\mathrm{i}}
\bmdefine\bj{\mathrm{j}}
\bmdefine\bk{\mathrm{k}}
\bmdefine\bl{\mathrm{l}}
\bmdefine\bm{\mathrm{m}}
\bmdefine\bn{\mathrm{n}}
\bmdefine\bo{\mathrm{o}}
\bmdefine\bp{\mathrm{p}}
\bmdefine\bq{\mathrm{q}}
\bmdefine\br{\mathrm{r}}
\bmdefine\bs{\mathrm{s}}
\bmdefine\bt{\mathrm{t}}
\bmdefine\bu{\mathrm{u}}
\bmdefine\bv{\mathrm{v}}
\bmdefine\bw{\mathrm{w}}
\bmdefine\bx{\mathrm{x}}
\bmdefine\by{\mathrm{y}}
\bmdefine\bz{\mathrm{z}}

\bmdefine\balpha{\alpha}
\bmdefine\bbeta{\beta}
\bmdefine\bgamma{\gamma}
\bmdefine\bdelta{\delta}
\bmdefine\blambda{\lambda}
\bmdefine\btheta{\theta}
\bmdefine\bphi{\phi}
\bmdefine\bvarphi{\varphi}
\bmdefine\bxi{\xi}
\bmdefine\bzeta{\zeta}
\bmdefine\boldeta{\eta}
\bmdefine\bpi{\pi}
\bmdefine\bnu{\nu}
\bmdefine\bmu{\mu}
\bmdefine\brho{\rho}
\bmdefine\bomega{\omega}
\bmdefine\bOmega{\ensuremath{\Omega}}
\bmdefine\bvarepsilon{\varepsilon}
\bmdefine\bDelta{\ensuremath{\Delta}}
\bmdefine\bTheta{\ensuremath{\Theta}}
\bmdefine\bSigma{\ensuremath{\Sigma}}
\bmdefine\bPsi{\ensuremath{\Psi}}
\bmdefine\bLambda{\ensuremath{\Lambda}}
\bmdefine\bzero{0}
\bmdefine\bone{1}
\bmdefine\binfty{\infty}

\newcommand{\indicator}{\mathbbm{1}}


% In python 
% print_iterable(['\\newcommand{\\'+'{0}cal'.format(x)+'}{\\mathcal{'+x+'}}' 
% for x in string.ascii_uppercase])
\newcommand{\Acal}{\mathcal{A}}
\newcommand{\Bcal}{\mathcal{B}}
\newcommand{\Ccal}{\mathcal{C}}
\newcommand{\Dcal}{\mathcal{D}}
\newcommand{\Ecal}{\mathcal{E}}
\newcommand{\Fcal}{\mathcal{F}}
\newcommand{\Gcal}{\mathcal{G}}
\newcommand{\Hcal}{\mathcal{H}}
\newcommand{\Ical}{\mathcal{I}}
\newcommand{\Jcal}{\mathcal{J}}
\newcommand{\Kcal}{\mathcal{K}}
\newcommand{\Lcal}{\mathcal{L}}
\newcommand{\Mcal}{\mathcal{M}}
\newcommand{\Ncal}{\mathcal{N}}
\newcommand{\Ocal}{\mathcal{O}}
\newcommand{\Pcal}{\mathcal{P}}
\newcommand{\Qcal}{\mathcal{Q}}
\newcommand{\Rcal}{\mathcal{R}}
\newcommand{\Scal}{\mathcal{S}}
\newcommand{\Tcal}{\mathcal{T}}
\newcommand{\Ucal}{\mathcal{U}}
\newcommand{\Vcal}{\mathcal{V}}
\newcommand{\Wcal}{\mathcal{W}}
\newcommand{\Xcal}{\mathcal{X}}
\newcommand{\Ycal}{\mathcal{Y}}
\newcommand{\Zcal}{\mathcal{Z}}

%%%%%%%%%%%%%%%%%%%%%%%%%%%% MSc operators %%%%%%%%%%%%%%%%%%%%%%%%%%%%
\DeclareMathOperator{\E}{\mathbb{E}} %used in gaussian processes
\newcommand{\Seiz}{\ensuremath{\mathbb{S}}}
\newcommand{\EEG}{\ensuremath{\mathbb{E}}}
\newcommand{\Seizt}{\ensuremath{\mathbb{S}_t}}
\newcommand{\EEGt}{\ensuremath{\mathbb{E}_t}}

%%%%%%%%%%%%%%%%%%%%%%%%%%%% Add to PDF to TOC %%%%%%%%%%%%%%%%%%%%%%%%%%%%
% https://tex.stackexchange.com/a/88101
%   
%---------------------------- begin macro for including a PDF document
% includepdf syntax:
%     addtotoc={⟨page number⟩,⟨section⟩, ⟨level⟩,⟨heading⟩,⟨label⟩}
%     addtolist={⟨page number⟩,⟨type⟩,⟨heading⟩,⟨label⟩}
%   \IncludeMyPDF
%   {1} %  page number to be included
%   {0.9} % scale
%   {true} %   landscape = true or false
%   {false} %  turn = true or false
%   {subsection,2} % level in TOC: section, subsection, subsubsection + level 1,2,3
%   {TitleTOC} %  heading for TOC / list 
%   {Label} %   label: label-toc-#7, label-list-#7, #7-target for hyperlinks
%   {table} %   addtolist = table or figure
%   {mindmaps.pdf} %  file

\newcommand{\IncludeMyPDF}[9]{%
\newpage\hypertarget{#7-target}
{\includepdf[pages={#1},nup=1x1,
    scale=#2,landscape=#3,turn=#4,
    pagecommand={\thispagestyle{empty}},
    addtotoc={#1,#5,#6,label-toc-#7},
    addtolist={#1,#8,#6,label-list-#7}]
{#9}}}

\newcommand{\IncludeMyPDFinReverse}[9]{%
\newpage\hypertarget{#7-target}
{\includepdf[pages={#1},nup=1x1,
    scale=#2,landscape=#3,turn=#4,
    pagecommand={\thispagestyle{empty}},
    addtotoc={#1,#5,#6,label-toc-#7},
    addtolist={#1,#8,#6,label-list-#7},
    pages=last-1]
{#9}}}


% %DIF PREAMBLE EXTENSION ADDED BY LATEXDIFF
%DIF UNDERLINE PREAMBLE %DIF PREAMBLE
\RequirePackage[normalem]{ulem} %DIF PREAMBLE
\RequirePackage{color}
% \definecolor{RED}{rgb}{1,0,0}
% \definecolor{BLUE}{rgb}{0,0,1} %DIF PREAMBLE

\colorlet{RED}{red!80!black}
\colorlet{BLUE}{blue!90!black}

\providecommand{\DIFadd}[1]{{\protect\color{BLUE}\uwave{#1}}} %DIF PREAMBLE
\providecommand{\DIFdel}[1]{{\protect\color{RED}\sout{#1}}}                      %DIF PREAMBLE
%DIF SAFE PREAMBLE %DIF PREAMBLE
\providecommand{\DIFaddbegin}{} %DIF PREAMBLE
\providecommand{\DIFaddend}{} %DIF PREAMBLE
\providecommand{\DIFdelbegin}{} %DIF PREAMBLE
\providecommand{\DIFdelend}{} %DIF PREAMBLE
\providecommand{\DIFmodbegin}{} %DIF PREAMBLE
\providecommand{\DIFmodend}{} %DIF PREAMBLE
%DIF FLOATSAFE PREAMBLE %DIF PREAMBLE
\providecommand{\DIFaddFL}[1]{\DIFadd{#1}} %DIF PREAMBLE
\providecommand{\DIFdelFL}[1]{\DIFdel{#1}} %DIF PREAMBLE
\providecommand{\DIFaddbeginFL}{} %DIF PREAMBLE
\providecommand{\DIFaddendFL}{} %DIF PREAMBLE
\providecommand{\DIFdelbeginFL}{} %DIF PREAMBLE
\providecommand{\DIFdelendFL}{} %DIF PREAMBLE
%DIF LISTINGS PREAMBLE %DIF PREAMBLE
\RequirePackage{listings} %DIF PREAMBLE
\RequirePackage{color} %DIF PREAMBLE
\lstdefinelanguage{DIFcode}{ %DIF PREAMBLE
%DIF DIFCODE_UNDERLINE %DIF PREAMBLE
  moredelim=[il][\color{RED}\sout]{\%DIF\ <\ }, %DIF PREAMBLE
  moredelim=[il][\color{BLUE}\uwave]{\%DIF\ >\ } %DIF PREAMBLE
} %DIF PREAMBLE
\lstdefinestyle{DIFverbatimstyle}{ %DIF PREAMBLE
	language=DIFcode, %DIF PREAMBLE
	basicstyle=\ttfamily, %DIF PREAMBLE
	columns=fullflexible, %DIF PREAMBLE
	keepspaces=true %DIF PREAMBLE
} %DIF PREAMBLE
\lstnewenvironment{DIFverbatim}{\lstset{style=DIFverbatimstyle}}{} %DIF PREAMBLE
\lstnewenvironment{DIFverbatim*}{\lstset{style=DIFverbatimstyle,showspaces=true}}{} %DIF PREAMBLE
%DIF END PREAMBLE EXTENSION ADDED BY LATEXDIFF
