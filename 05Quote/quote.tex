%%%%%%%% ageless quote %%%%%%%%
% \epigraph{
%     It is not sufficient merely to observe; \newline
%     we must use our observations, \newline
%     and for that purpose we must generalise.\newline
%     This is what has always been done, \newline
%     only as the recollection of past errors has made man more and more circumspect, \newline
%     he has observed more and more \newline
%     and generalised less and less. \newline
%     \newline
%     Every age has scoffed at its predecessor, \newline
%     accusing it of having generalised too boldly and too naïvely. \newline
%     Descartes used to commiserate the Ionians. Descartes in his turn makes us smile, \newline
%     and no doubt some day our children will laugh at us. \newline
%     \newline
%     Is there no way of getting at once to the gist of the matter, \newline
%     and thereby escaping the raillery which we foresee? \newline
%     Cannot we be content with experiment alone? \newline
%     \newline
%     No, that is impossible; \newline
%     that would be a complete misunderstanding \newline
%     of the true character of science. \newline
%     \newline
%     The man of science \newline
%     must work with method. \newline
%     Science is built up of facts, \newline
%     as a house is built of stones; \newline
%     but an accumulation of facts is no more a science \newline
%     than a heap of stones is a house. \newline
%     Most important of all, \newline
%     the man of science \newline
%     must exhibit foresight.
% }{ \textit{Science and Hypothesis\\
%     Henri Poincaré}
% }

\epigraph{"The theory of probabilities is at bottom nothing but common sense reduced to calculus"
}{\textit{Essai Philosophique sur les Probabilités} \\ 
    Pierre-Simon Laplace}
