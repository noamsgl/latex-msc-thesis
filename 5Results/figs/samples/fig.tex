\begin{figure}[htb]
    \centering
    \begin{subfigure}[b]{0.5\floatwidth}
        \includegraphics[width=\textwidth]{5Results/figs/samples/double_interictal_sample.png}
    \end{subfigure}
    \hfill    
    \begin{subfigure}[b]{0.5\floatwidth}
        \includegraphics[width=\textwidth]{5Results/figs/samples/double_interictal_pvalue.png}
    \end{subfigure}
    \\
    \begin{subfigure}[b]{0.5\floatwidth}
        \includegraphics[width=\textwidth]{5Results/figs/samples/double_ictal_sample.png}
    \end{subfigure}
    \hfill
    \begin{subfigure}[b]{0.5\floatwidth}
        \includegraphics[width=\textwidth]{5Results/figs/samples/double_ictal_pvalue.png}
    \end{subfigure}
    \Caption{Anomaly detection}{
	Interictal and ictal EEG samples, along with the calculated p-values for each clip. As can be seen, the interictal clip has a p-value of 0.4355 which indicates that it is in the normal distribution. On the other hand, the ictal clip has an extreme p-value of 0.9787 which indicates that it is a rare event, thus in a patient with epilepsy it is likely to be a seizure.
	\protect \NS[inline]{remake example likelihood estimation figures}
    }
    \label{fig:5results:samples}
\end{figure}