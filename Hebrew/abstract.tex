\chapter*{תקציר}


מודלים חישוביים של התקפים באפילפסיה מ-GEE יאפשרו למכשירים לבישים להתריע בפני משתמשים לפני תחילת התקף. שיטות זיהוי דפוסים מפוקחות השיגו תוצאות מבטיחות בהבחנה בין התקפים מוקדמים לתפקוד מוח תקין, כפי שמיוצג על ידי וקטורים של GEE. מכיוון שסימון התקפים אינו זמין בקנה מידה גדול, הצלחתן של טכניקות זיהוי דפוסים אלו תהיה מוגבלת ככל שהנתונים יגדלו.

אנו מציעים חלופה בייסיאנית בלתי-מפוקחת למסווגים לזיהוי התקפים וחיזוי. שיטה זו חסכונית בסימונים ומשיגה ציון 0.88 COR-UA עם אפס תוויות במשימת זיהוי התקפים. גרסה בפיקוח חלש המוטה למקצבים הצירקדיים משפרת את הזיהוי אצל כלבים. השיטה שלנו מורכבת משני שלבים: ראשית, התפלגות ה-GEE משוערכת ומקצה לרצפים חריגים סבירות גבוהה יותר להתקפים. לאחר מכן, נוסף ידע אפריורי המבוסס על משתני הזמן לשיפורים מונחים.