% =========================
\appendix
\chapter{GP embedding: training details}
\label{apx:GPTraining}

% =========================
For dimensionality reduction of an observed EEG segment, we performed exact inference of the GP parameters maximizing the observation likelihood, using the GPyTorch and PyTorch Lightning frameworks \shortcites{gardner2018gpytorch} \cite{gardner2018gpytorch, Falcon_PyTorch_Lightning_2019}.

More formally, in fitting the Gaussian processes to the EEG samples we carried out exact Type-II Maximum Likelihood Estimation for each sample. This means optimizing the model's hyperparameters (mean module, covariance module, etc.) w.r.t maximization of the \emph{marginal log likelihood} (MLL) of the given data $\mathbf{E, t}$:

\begin{align}
    \theta \gets \argmax_\theta p_f(\mathbf{E} \mid \mathbf{t}) = \int p(\mathbf{E} \mid f(\mathbf{t})) p(f(\mathbf{t}) \mid \mathbf{t}) df
\end{align}

where $f \sim \mathcal{GP}(\mu, K)$ is the modeled signal before adding the homescedastic Gaussian noise, and $\theta$ is the set of parameters to be optimized. See code for implementation details.

\begin{table}[h]
\begin{tabular}{ |p{3cm}||p{5cm}||p{5cm}| }
 \hline
 \multicolumn{3}{|c|}{GP \& inference (training) configuration details} \\
 \hline
  &Parameter& Value \\
 \hline
\multirow{5}{3cm}{GP params} & mean module & zero mean\\
&covariance module& scaled Matérn-1.5 kernel\\
&task covariance rank& 1 \\
&number of tasks& 2   \\
&likelihood (noise model)&Gaussian (homoscedastic)\\
\hline
\multirow{4}{3cm}{Training params}& optimizer &Adam\\
&learning rate& 0.01 \\
&max. number of epochs & 1000\\
&patience (early stopping)&8 \\
 \hline
\end{tabular}
\caption{The GP parameters and training parameters used in our experiments.}
\label{table:GPTraining}
\end{table}
